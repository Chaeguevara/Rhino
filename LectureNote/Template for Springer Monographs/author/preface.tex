%%%%%%%%%%%%%%%%%%%%%%preface.tex%%%%%%%%%%%%%%%%%%%%%%%%%%%%%%%%%%%%%%%%%
% sample preface
%
% Use this file as a template for your own input.
%
%%%%%%%%%%%%%%%%%%%%%%%% Springer %%%%%%%%%%%%%%%%%%%%%%%%%%

\preface

%% Please write your preface here
본 책에서는 파라메트릭(parametric) 또는 procedural 모델링이라고 불리는 방법을 수행하는 그래스호퍼(grasshopper)\footnote{https://www.grasshopper3d.com/}를 이용해 제작가능한 수준의 3d 모델을 만들고, 데이터를 추출하는 법에 대해서 다룹니다. 그렇기 때문에 본 책의 내용은 디자인보다는 좀더 공학에 가깝고, 수학과 같이 정해진 답을 찾는 과정에 가깝습니다. 어떤분들은 이런 내용을 싫어할지 모르지만, 여러분의 작업물이 paper architec가 아닌 실제로 지어지기 위해서는 굉장히 중요한 내용입니다.

많은 분들은 \textbf{그래스호퍼} 하면 마치 비정형 건물을 다뤄야 할 것이라고 생각하지만, 파라메트릭 또는 procedural라는 단어에서 보듯 파라미터(parameter)만 변경해서 재사용할 수 있는 작업절차를 만들고, 이후에 입력값인 파라미터만 변경해 같은 작업절차를 반복 사용하는 도구이기 때문에 \textbf{비정형을 만드는 도구가 아닌 작업을 효율적으로 만들어 주는 도구입니다}.

따라서 본 책에서는 크게 두가지 주제를 다룹니다. 

\textbf{첫번째}는 지오메트리(geometry)를 조정해, 제작에 용이한 지오메트리로 변경하는 과정을 다룹니다. 여기서 제작에 용이한 지오메트리는 전개가능한 형상(developable surface)을 뜻하고, 이를 다시 풀어쓰면 변형(deformation)없이 2d 전개도를 만들 수 있는 3d 형상을 뜻합니다. 

\textbf{두번째}는 만들어진 지오메트리로 부터 제작가능한 데이터(data)를 추출하는 작업에 대해서 다룹니다. 실제로 공장기계(manufacturing machine)에 따라 요구하는 데이터는 다릅니다. 대표적으로는 2d 전개도가 있지만, 곡면 패널의 경우 2d 전개도와 더불어 각 전개도에 대응되는 R(radius)값을 요구하기도 합니다. 따라서 이러한 주제도 다룹니다.

위 두가지를 수행하기 위해 필요한 플러그인 사용법과 간단한 스크립팅 내용도 포함합니다.
\vspace{\baselineskip}
\begin{flushright}\noindent
Seoul, Korea,\hfill {\it 채희진}\\
Dec 2022\hfill {\it ??}\\
\end{flushright}


